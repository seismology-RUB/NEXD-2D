\chapter{Output}
\label{chap:output}
	By default, the output of NEXD 2D is stored in a subdirectory to the simulation directory called \texttt{out}. If the script ``process.sh'' is used, the parameter files used for the current simulation are copied to the \texttt{out} folder for the users convenience. That way, the user is always able to reconstruct what parameters were used to create this simulation. If a full waveform inversion is applied, output files are also saved in \texttt{inversion} and \texttt{adjoint}.
	
	\section{Files created by \texttt{mesher}}
	\label{sec:mesher}
		For every simulation, files containing information of parts of the mesh called ``meshVar0000001'' are created. The number denotes the running number of the processor the file was created for.
		
		If stations are placed in the model, files are created that contain the information on the location of the station, where it is located in the mesh and in which processor the receiver is located. These files are called for example ``recVar000013''.
		
		If fractures are included, files that map the slip interfaces to their elements called for example ``elementToLSI000032'' will be created as well.
		
		A number of visualisation files are created. They are listed in table \ref{tab:vtk_files}.
		\begin{table}[ht]
   			\centering
    		\caption{List of vtk-files given as output.}
    		\label{tab:vtk_files}
    		\begin{tabular}{@{} l p{9cm} @{}}
        		\toprule
        		\multicolumn{1}{c}{Name} & Content\\ 
        		\midrule
        		mesh.vtk & triangular mesh \\
        		rec.vtk & receiver positions \\    
       			src.vtk & source positions \\
        		rho\_model.vtk & density model \\
        		vp\_model.vtk & p-wave velocity model (if not a poroelastic simulation) \\
        		vs\_model.vtk & s-wave velocity model (if not a poroelastic simulation) \\
        		vmax\_model.vtk & maximum wave velocity model (if poroelastic simulation) \\
        		vmin\_model.vtk & minimum wave velocity model (if poroelastic simulation) \\
        		\bottomrule
    		\end{tabular}
		\end{table}

	\section{Files created by \texttt{solver}}
	\label{sec:solver}
		The output from \texttt{solver} varies with the parameters selected in the parameter files. Seismograms will be created, if stations are placed in the model and the correct parameters have been set. Binary files to generate visualisations of the wave-field are only generated if the appropriate parameters are set in the ``movie'' portion of the parameter file (see section \ref{subsubsec:moviepar}).
		
		\subsection{Seismograms}
			Seismograms are saved as plain ASCII text files with two columns of data: The first column contains the time-axis and the second column contains the respective data. Three types of data are available: displacement, particle velocity and acceleration. Each type has its own file distinguished by the file name. The file-name is constructed in the following way:
			\begin{center}%[4cm]{4cm}% 1em left, 2em right
				\emph{seismo.(component).(number of the station).sd(type)}
			\end{center}
			The component is ``x'', ``z'' or ``p'' if it is pressure data. If the parameter div and/or curl are enabled this may also be ``r'' or ``t'' respectively, where ``r'' is the radial and ``t'' the tangential component. The number of the station is a seven-digit running number starting from 0000001 to the total number of stations (cf., nrec in sec.~\ref{subsec:stations}) placed in the model. The file extension .sd* tells the user which kind of data is contained in the file: ``a'' for acceleration, ``v'' for (solid) particle velocity, ``v1'' and ``v2'' for fluid particle velocity, ``p1'' and ``p2'' for fluid pressure, and ``u'' for displacement.
			
			\medskip
			\fcolorbox{green}{white}{\parbox{\textwidth-2\fboxsep-2\fboxrule}{\textbf{Example}: Given these conventions, a seismogram for the x-component of station 23 containing displacement data is named: \emph{seismo.x.0000023.sdu}.}}
			
		\subsection{Binary file to create vtk files}
		\label{subsec:binvtk}
			The exact amount and type of binary files created by \texttt{solver} depends on the choice of parameters. First of all, the parameter ``movie'' needs to be enabled so that any file may be created. If that is the case, a file is created for each time-step matching the ``frame'' parameter per processor.
			
			\medskip
			\fcolorbox{green}{white}{\parbox{\textwidth-2\fboxsep-2\fboxrule}{\textbf{Example}: If frame = 100, a file will be created for each 100th time-step. }}
			
			\medskip
			Additionally, the user can enable the creation of binary files for acceleration, particle velocity and/or displacement. If either is selected, files will be created that contain the x-component, the z-component and the norm of the selected property, respectively. The files are named in the following way:
			\begin{center}%[3cm]{3cm}% 1em left, 2em right
				\emph{moviedata.(type)(component).(number of the processor)\_it(time-step).bin}
			\end{center} 
			Here, type is ``a'', ``v'', ``v1'', ``v2'', ``u'', ``p1'', ``p2'', or ``stress'', the component is ``x'', ``z'' or ``norm''. In case of the norm, type and component are reversed. The number of processor is a six-digit number of the processor that created the file and time-step is a seven-digit number that describes the time-step that the file represents.

			\medskip
			\fcolorbox{green}{white}{\parbox{\textwidth-2\fboxsep-2\fboxrule}{\textbf{Example}: Given these conventions, a movie-binary file for the x-component of the particle velocity created in processor 23 at time-step 4000 is named: \emph{moviedata\_vx\_000023\_it0004000.bin}. If case of the norm, but otherwise identical parameters the file is named: \emph{moviedata\_normV\_000023\_it0004000.bin}.}}
			
		\subsection{Output during an inversion}
		\label{subsec:outinv}
		
		During an inversion process a lot of data is produced. After each single forward or adjoint simulation the standard output described above is saved in folder \texttt{out}. Additionally, the program \texttt{movie} (see section \ref{sec:movieout}) is called automatically. All these files will be overwritten by the next simulation since they are of minor interest for the inversion process. However, it is possible to check them during the inversion.  
		
		\subsubsection{Output in \texttt{adjoint}}
			Here, binary files (*.bin files) to produce vtk files for the misfit gradient similar to the movie data (see section \ref{subsec:binvtk}) are saved. A vtk file of the misfit gradient for each iteration step is automatically produced after the iteration step for each single source. The file name of the misfit gradient vtk files are named as follows: 
		
			\begin{center}%[3cm]{3cm}% 1em left, 2em right
				\emph{movie\_element\_K(type)\_src(number of source)\_it(iteration step).vtk}
			\end{center}
			
			Type is either ``vp'' or ``vs'' describing P- or S-wave velocity, respectively. The number of source is a three-digit number associated with the source number in the \texttt{source} file and iteration step is a seven-digit number describing simply the iteration step in which the gradient was used. 
			
			Additionally to these files the source time function of all adjoint sources are saved. Their names are built as follows:
			\begin{center}%[3cm]{3cm}% 1em left, 2em right
				\emph{adj\_stf(number of station)iter(iteration step)src(number of source)}
			\end{center}
			
			The number of the station is a seven-digit number corrensponding to the station number in the \texttt{station} file, the iteration number is simply the running number of the iterations of the inversion procedure, and the number of the source corresponds to the source number in the \texttt{source} file. 
		\subsubsection{Output in \texttt{inversion}}
			In this directory the four files:
			
			\begin{itemize}
				\item alpha
				\item frequency
				\item misfit
				\item time
			\end{itemize}						
			
			can be found. All four files show the iteration step in the first column and in the second column they show the used step width (alpha), the cutoff frequency of the applied low-pass filter (frequency), the misfit corresponding to the model used at the beginning of the iteration step (misfit), and the total computational time (time).
			
			\medskip
			Additionally, here are binary files (*.bin files)  saved that are used to create vtk files for the search direction and the velocity model after the model change for every iteration step. They are named as follows: 
			
			\begin{center}%[3cm]{3cm}% 1em left, 2em right
				\emph{movie\_element\_cvp\_it(iteration step).vtk}\\
				\emph{movie\_element\_cvs\_it(iteration step).vtk}\\
				\emph{movie\_element\_model\_vp\_it(iteration step).vtk}\\
				\emph{movie\_element\_model\_vs\_it(iteration step).vtk}
			\end{center}
			
			The first two show the search direction for the P- and S-wave velocity, respectively, and the last two show the current velocity model after the model change of the specific iteration step is applied. The variable iteration step is a seven-digit number of the running iteration step during the inversion procedure. 
			
			Seismograms for each station and source are also saved here. They have the structure:
			
			\begin{center}%[3cm]{3cm}% 1em left, 2em right
				\emph{seismo.(component).(station number).sdu(iteration number)run1src(source number)}
			\end{center}
			
			The component is either ``x'' or ``z''. The station number is a seven-digit number corresponding the station number in \texttt{station}, the iteration step is a three-digit number of the running iteration step during the inversion procedure, and the source number corresponds to the source number in the \texttt{source} file. 
			
			For every measurement file (see section \ref{sec:ininvfiles}) files with an appendix ``filter(frequency)'' are created. They contain the filtered measurements while frequency in the appendix is a four-digit number for the cutoff frequency of the low-pass filter in Hertz.
			
		\subsubsection{Temporal wave field files}
		
		During the inversion process large files containing the wave field of the forward simulation need to be saved to disk. These files take a lot of storage capacity and a low disk write rate can slow down the simulation strongly. By default, these files are saved in the \texttt{inversion} folder. If your local \texttt{inversion} folder is not able to handle such files, please change the variable ``temppath'' in the file \texttt{src/constants.h} to a suitable directory and recompile the complete code. Please have in mind that each processor will use this directory from its own point of view. If your simulation is distributed over processors of different computers on a large cluster every processor will call the directory from the computer it is running on.
			
	\section{Files created by \texttt{movie}}
	\label{sec:movieout}
		There are two possible outputs: 
		\begin{itemize}
			\item trimesh: Creates file, where the average  over an element is plotted.
			\item point: Creates files, which contain the information on each grid point.
		\end{itemize}
		Both variants can be used independently and together.
					
		\subsection{trimesh files}
			This version of the wave filed output is enabled by selecting the parameter ``save\_movie\_trimesh''. The files are named in the following way:
			\begin{center}%[3cm]{3cm}% 1em left, 2em right
				\emph{movie\_element\_(type)(component)\_it(framenumber).vtk}
			\end{center}
			The part of the filename containing type and component follows the same pattern as above. The frame-number is a seven-digit number that increases sequentially.

			\medskip
			\fcolorbox{green}{white}{\parbox{\textwidth-2\fboxsep-2\fboxrule}{\textbf{Example}: Given these conventions, the $4^{th}$ movie-vtk for the x-component of the particle velocity is called: \emph{movie\_element\_vx\_it0000004.vtk}. In case of the norm, but otherwise identical parameters the file is called: \emph{movie\_element\_norm\_v\_it0000004.vtk}.}}
		
		\subsection{points files}
			This version of the wave-field output takes significantly more computational time as well as space on the hard-disk, but contains more detailed information. It is enabled by selecting the parameter ``save\_movie\_points''. The files are named in the following way:
			\begin{center}%[3cm]{3cm}% 1em left, 2em right
				\emph{movie\_points\_(type)(component)\_it(framenumber).vtk}
			\end{center}
			The part of the filename containing type and component follows the same pattern as above. The frame-number is a seven-digit number that increases sequentially.
			
			\medskip
			\fcolorbox{green}{white}{\parbox{\textwidth-2\fboxsep-2\fboxrule}{\textbf{Example}: Given these conventions, the $4^{th}$ movie-vtk for the x-component of the particle velocity is named: \emph{movie\_points\_vx\_it0000004.vtk}. In case of the norm, but otherwise identical parameters the file is named: \emph{movie\_points\_norm\_v\_it0000004.vtk}.}}